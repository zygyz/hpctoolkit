%% $Id: hpctraceviewer.tex 3428 2011-02-22 21:31:42Z tallent $

%%%%%%%%%%%%%%%%%%%%%%%%%%%%%%%%%%%%%%%%%%%%%%%%%%%%%%%%%%%%%%%%%%%%%%%%%%%%%
%%%%%%%%%%%%%%%%%%%%%%%%%%%%%%%%%%%%%%%%%%%%%%%%%%%%%%%%%%%%%%%%%%%%%%%%%%%%%

\documentclass[english]{article}
\usepackage[latin1]{inputenc}
\usepackage{babel}
\usepackage{verbatim}

%% do we have the `hyperref package?
\IfFileExists{hyperref.sty}{
   \usepackage[bookmarksopen,bookmarksnumbered]{hyperref}
}{}

%% do we have the `fancyhdr' or `fancyheadings' package?
\IfFileExists{fancyhdr.sty}{
\usepackage[fancyhdr]{latex2man}
}{
\IfFileExists{fancyheadings.sty}{
\usepackage[fancy]{latex2man}
}{
\usepackage[nofancy]{latex2man}
\message{no fancyhdr or fancyheadings package present, discard it}
}}

%% do we have the `rcsinfo' package?
\IfFileExists{rcsinfo.sty}{
\usepackage[nofancy]{rcsinfo}
\rcsInfo $Id: hpctraceviewer.tex 3428 2011-02-22 21:31:42Z tallent $
\setDate{\rcsInfoLongDate}
}{
\setDate{ 2011/02/22}
\message{package rcsinfo not present, discard it}
}

\setVersionWord{Version:}  %%% that's the default, no need to set it.
\setVersion{=PACKAGE_VERSION=}

%%%%%%%%%%%%%%%%%%%%%%%%%%%%%%%%%%%%%%%%%%%%%%%%%%%%%%%%%%%%%%%%%%%%%%%%%%%%%
%%%%%%%%%%%%%%%%%%%%%%%%%%%%%%%%%%%%%%%%%%%%%%%%%%%%%%%%%%%%%%%%%%%%%%%%%%%%%

\begin{document}



\begin{Name}{1}{hpctraceviewer}{The HPCToolkit Performance Tools}{The HPCToolkit Performance Tools}{hpctraceviewer:\\ Interactive Presentation of Program Traces}

The Java-based \Prog{hpctraceviewer} interactively presents dynamic behavior of a program. 

\end{Name}

%%%%%%%%%%%%%%%%%%%%%%%%%%%%%%%%%%%%%%%%%%%%%%%%%%%%%%%%%%%%%%%%%%
\section{Synopsis}

Command-line usage:\\
\SP\SP\SP\Prog{hpctraceviewer} \oOpt{options} \oOpt{hpctoolkit-database}

GUI usage:\\
\SP\SP\SP Launch \File{hpctraceviewer} and open the Experiment database \oOpt{hpctoolkit-database}.


%%%%%%%%%%%%%%%%%%%%%%%%%%%%%%%%%%%%%%%%%%%%%%%%%%%%%%%%%%%%%%%%%%
\section{Description}

The Java-based \Prog{hpctraceviewer} interactively presents program traces in a top-down fashion.
Since Experiment databases are self-contained, they may be relocated from a cluster for visulization on a laptop or workstation.

%%%%%%%%%%%%%%%%%%%%%%%%%%%%%%%%%%%%%%%%%%%%%%%%%%%%%%%%%%%%%%%%%%
\section{Arguments}

\begin{Description}
\item[\Arg{hpctoolkit-database}] An HPCToolkit Experiment database, which is the result of executing \Prog{hpcprof}, \Prog{hpcprof-mpi} or \Prog{hpcprof-flat}.
\end{Description}

%Default values for an option's optional arguments are shown in \{\}.

\subsection{Options}

\begin{Description}

\item[\Opt{-consolelog}]
Send log entries to a console in addition to a log file.

\item[\Opt{-debug}]
Log additional information about plug-in dependency problems.

\end{Description}


%%%%%%%%%%%%%%%%%%%%%%%%%%%%%%%%%%%%%%%%%%%%%%%%%%%%%%%%%%%%%%%%%%
\section{Detailed Description}

\subsection{Views}

\Prog{hpctraceviewer} comprises of three different parts. 


\begin{itemize}
\item Trace view (left, top):
  This is hpctraceviewer's primary view.
  This view, which is similar to a conventional process/time (or space/time) view, shows time on the horizontal axis and process (or thread) rank on the vertical axis; time moves from left to right.
  Compared to typical process/time views, there is one key difference.
  To show call path hierarchy, the view is actually a user-controllable slice of the process/time/call-path space.
  Given a call path depth, the view shows the color of the currently active procedure at a given time and process rank.
  (If the requested depth is deeper than a particular call path, then hpctraceviewer simply displays the deepest procedure frame and, space permitting, overlays an annotation indicating the fact that this frame represents a shallower depth.)
  hpctraceviewer assigns colors to procedures based on (static) source code procedures.
  Although the color assignment is currently random, it is consistent across the different views.
  Thus, the same color within the Trace and Depth Views refers to the same procedure.
  The Trace View has a white crosshair that represents a selected point in time and process space.
  For this selected point, the Call Path View shows the corresponding call path.
  The Depth View shows the selected process.

\item Depth view (left, bottom):
  This is a call-path/time view for the process rank selected by the Trace view's crosshair.
  Given a process rank, the view shows for each virtual time along the horizontal axis a stylized call path along the vertical axis, where `main' is at the top and leaves (samples) are at the bottom.
  In other words, this view shows for the whole time range, in qualitative fashion, what the Call Path View shows for a selected point.
  The horizontal time axis is exactly aligned with the Trace View's time axis; and the colors are consistent across both views.
  This view has its own crosshair that corresponds to the currently selected time and call path depth.

\item Summary view (same location as depth view on the left-bottom part):
  The view shows for the whole time range dislayed, the proportion of each subroutine in a certain time.
  Similar to Depth view, the time range in Summary reflects to the time range in the Trace view. 

\item Call view (right, top):
  This view shows two things: (1) the current call path depth that defines the hierarchical slice shown in the Trace View; and (2) the actual call path for the point selected by the Trace View's crosshair.
  (To easily coordinate the call path depth value with the call path, the Call Path View currently suppresses details such as loop structure and call sites; we may use indentation or other techniques to display this in the future.)

\item Mini map view (right, bottom):
  The Mini Map shows, relative to the process/time dimensions, the portion of the execution shown by the Trace View.
  The Mini Map enables one to zoom and to move from one close-up to another quickly.

\end{itemize}


% ===========================================================================

\subsection{Trace view}

Trace view is divided into two parts: the top part which contains \emph{action pane} and the \emph{information pane}, and the main view which displays the traces. 

The buttons in the action pane are the following:
\begin{itemize}

\item Home : Resetting the view configuration into the original view, i.e., viewing traces for all times and processes.
\item Horiontal zoom in / out : Zooming in/out the time dimension of the traces. 
\item Vertical zoom in / out : Zooming in/out the process dimension of the traces.
\item Navigation buttons : Navigating the trace view to the left, right, up and bottom, respectively. It is also possible to navigate with the arrow keys in the keyboard. Since Trace view does not support scrool bars, the only way to navigate is through navigation buttons (or arrow keys).
\item Undo : Canceling the action of zoom or navigation and returning back to the previous view configuration.
\item Redo : Redoing of previously undo change of view configuration.
\item save  / a view configuration : Saving/loading a saved view configuration. 
A view configuration file contains the information of the current dimension of time and process, the depth and the position of the crosshair. 
It is recommended to store the view configuration file in the same directory as the database to ensure that the view configuration file matches well with the database since the file does not store which database it is associated with. 
Although it is possible to open a view configuration file which is associated from different database, it is highly not recommended since each database has different time/process dimensions and depth.


\end{itemize}

The information pane contains some information concerning the range status of the current displayed data.
\begin{itemize}
 \item Time Range. The information of current time-range (horizontal) dimension. 
 \item Process Range. The information of current process-range (vertical) dimension. 
 \item Cross Hair. The information of current crosshair position in time and process dimensions. 
\end{itemize}


% ===========================================================================
% ===========================================================================
\subsection{Depth view}

Depthview shows all the call path for a certain time range [t_1,t_2]= \{t | t_1 <= t <= t_2\} in a specified process rank $p$. The content of Depth view is always consistent with the position of the cross-hair in Trace view.
For instance once the user clicks in process $p$ and time $t$, while the current depth of call path is $d$, then the Depth view's content is updated to display all the call path of process $p$ and shows its cross-hair on the time $t$ and the call path depth $d$.

On the other hand, any user action such as cross-hair and time range selection in Depth view will update the content within Trace view. Similarly, the selection of new call path depth in Call view invokes a new position in Depth view.

In Depth view a user can specify a new cross-hair time and a new time range.

\textbf{Specifying a new cross-hair time.} Selecting a new cross-hair time $t$ can be performed by clicking a pixel within Depth view. This will update the cross-hair in Trace view and the call path in Call view.

\textbf{Selecting a new time range.} Selecting a new time range [t_m,t_n]= \{t | t_m <= t <= t_n\} is performed by first clicking the position of $t_m$ and drag the cursor to the position of $t_n$. A new content in Depth view and Trace view is then updated. Note that this action will not update the call path in Call view since it does not change the position of the cross-hair.


% ===========================================================================
% ===========================================================================
\subsection{Summary view}

Summary view presents the proportion of number of calls of time $t$ across the current displayed rank of proces $p$. 
Similar to Depth view, the time range in Summary view is always consistent with the time range in Trace view.


% ===========================================================================
% ===========================================================================
\subsection{Call view}

This view lists the call path of process \textbf{p} and time \textbf{t} specified in Trace view and Depth view.
This view can show a call path from depth $0$ to the maximum depth, and the current depth is shown in the depth editor (located on the top part of the view).

In this view, the user can select the depth dimension by either typing the depth in the depth editor or selecting a procedure in the table of call path.

% ===========================================================================
% ===========================================================================
\subsection{Mini view}

The Mini view shows, relative to the process/time dimensions, the portion of the execution shown by the Trace view.
In Mini view, the user can select a new process/time (p_a,t_a),(p_b,t_b) dimensions by clicking the first process/time position (p_a,t_a) and then drag the cursor to the second position (p_b,t_b).
The user can also moving the current selected region to another region by clicking the white rectangle and drag it to the new place.



% ===========================================================================
% ===========================================================================

\section{Menus}
\Prog{hpctraceviewer} provides three main menus:
\begin{itemize}
 \item \textbf{File} menu which contains two sub menus:
 \begin{itemize}
   \item \textbf{Open database}: to load a database experiment directory. The directory has to contain \texttt{experiment.xml} (CCT and metric information) or \texttt{callpath.xml} (uniquely CCT information), and \texttt{*.hpctrace} or \texttt{experiment.mt} files which provide trace information.
   \item \textbf{Exit}: to quit the application.
 \end{itemize}
 \item \textbf{View} menu to enhance appearance which contains two sub menus:
 \begin{itemize}
   \item \textbf{Show debug info}: to enable/disabe the display of debugging information in the form of \textbf{`a(b)'} where \textbf{a} is the maximum depth (this number is shown if the current depth reaches the maximum depth) and \textbf{b} is the number of records on the trace view. 
The number of records can be useful to identify blocking procedures (such as I/O operations). Note: the numbers are displayed only if there's enough space in the process time line.
   \item \textbf{Using midpoint painting}: if checked, the trace painting will use {midpoint} painting algorithm. By using the later, for every samples $S1$ at time $T1$, $S2$ at time $T2$ and $S3$ at time $T3$, \Prog{hpctraceviewer} renders a block from $T1$ to $(T1+T2)/2$ to sample $S1$, and a block from from  $(T1+T2)/2$ to  $(T2+T3(/2$ for sample $S2$, and so forth. 
If the menu is not checked, then a simpler {rightmost} algorithm is used: it will render a block from  $T1$ to $T2$ for sample $S1$, and a block from $T2$ to  $T3$ for sample $S2$, and so forth.
   \item \textbf{Show procedure-color mapping}: to open a window which shows customized mapping between a procedure pattern and a color. \Prog{hpctraceviewer} allows users to customize assignment of a pattern of procedure names with a specific color.
   \item \textbf{Filter ranks}: to open a window for selecting which ranks should be displayed. Recall that a rank can be a process (e.g. MPI applications), a thread (OpenMP applications) or both (hybrid MPI and OpenMP applications). \Prog{hpctraceviewer} allows two types of filtering: either you specify which ranks {to show} or {to hide} (default is to hide). 
To add a pattern to filter, you need to click "add" button and type the pattern in the dialog box. To remove a pattern, you have to select the pattern to remove, and click "Remove" button. Finally, clicking to "Remove all" button will clear the list of patterns.
 \end{itemize}
 \item \textbf{Window} menu to manage the layout of the application. The menu only provide one sub menu:
 \begin{itemize}
  \item \textbf{Reset layout}: to reset the layout to the original one.
 \end{itemize}
\end{itemize}

\Prog{hpctraceviewer} also provides a context menu to save the current image of the view. 
This context menu is available is three views: trace view, depth view and summary view.
% ===========================================================================



%%%%%%%%%%%%%%%%%%%%%%%%%%%%%%%%%%%%%%%%%%%%%%%%%%%%%%%%%%%%%%%%%%
%\section{Arguments}

%%%%%%%%%%%%%%%%%%%%%%%%%%%%%%%%%%%%%%%%%%%%%%%%%%%%%%%%%%%%%%%%%%
%\section{Examples}

%%%%%%%%%%%%%%%%%%%%%%%%%%%%%%%%%%%%%%%%%%%%%%%%%%%%%%%%%%%%%%%%%%
%\section{Notes}

%%%%%%%%%%%%%%%%%%%%%%%%%%%%%%%%%%%%%%%%%%%%%%%%%%%%%%%%%%%%%%%%%%
\section{See Also}

\HTMLhref{hpctoolkit.html}{\Cmd{hpctoolkit}{1}}.

%%%%%%%%%%%%%%%%%%%%%%%%%%%%%%%%%%%%%%%%%%%%%%%%%%%%%%%%%%%%%%%%%%
\section{Version}

Version: \Version

%%%%%%%%%%%%%%%%%%%%%%%%%%%%%%%%%%%%%%%%%%%%%%%%%%%%%%%%%%%%%%%%%%
\section{License and Copyright}

\begin{description}
\item[Copyright] \copyright\ 2002-2020, Rice University.
\item[License] See \File{README.License}.
\end{description}

%%%%%%%%%%%%%%%%%%%%%%%%%%%%%%%%%%%%%%%%%%%%%%%%%%%%%%%%%%%%%%%%%%
\section{Authors}

\noindent
Rice University's HPCToolkit Research Group \\
Email: \Email{hpctoolkit-forum =at= rice.edu} \\
WWW: \URL{http://hpctoolkit.org}.

\LatexManEnd

\end{document}

%% Local Variables:
%% eval: (add-hook 'write-file-hooks 'time-stamp)
%% time-stamp-start: "setDate{ "
%% time-stamp-format: "%:y/%02m/%02d"
%% time-stamp-end: "}\n"
%% time-stamp-line-limit: 50
%% End:

